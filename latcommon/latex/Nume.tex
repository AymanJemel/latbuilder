% Copyright (c) 2012 Richard Simard, Pierre L'Ecuyer, Université de Montréal.
% 
% This file is part of Lattice Builder.
% 
% Lattice Builder is free software: you can redistribute it and/or modify
% it under the terms of the GNU General Public License as published by
% the Free Software Foundation, either version 3 of the License, or
% (at your option) any later version.
% 
% Lattice Builder is distributed in the hope that it will be useful,
% but WITHOUT ANY WARRANTY; without even the implied warranty of
% MERCHANTABILITY or FITNESS FOR A PARTICULAR PURPOSE.  See the
% GNU General Public License for more details.
% 
% You should have received a copy of the GNU General Public License
% along with Lattice Builder.  If not, see <http://www.gnu.org/licenses/>.

\defmodule {Nume}

This module implements a few useful mathematical functions.

%%%%%%%%%%%%%%%%%%%%%%%%%%%%%%%%%%%%%%%%%%%%%%%%%%%%%%%%%%%%%%
\bigskip\hrule
\code\hide
#ifndef NUME_H
#define NUME_H
\endhide

namespace LatCommon {

long lFactorial (int t);
\endcode
\tab
Calculates $t!$, the factorial of $t$.
\endtab
\code

double Digamma (double x);
\endcode
\tab Returns the value of the logarithmic derivative of the Gamma function
   $\psi(x) = \Gamma'(x) / \Gamma(x)$.
\endtab
\code

double BernoulliPoly (int n, double x);
\endcode
\tab Evaluates the Bernoulli polynomial $B_n(x)$ of degree $n$
  at $x$.
 The first Bernoulli polynomials are:
\begin{align}
B_0(x) &= 1 \nonumber \\
B_1(x) &= x - 1/2 \nonumber \\
B_2(x) &= x^2-x+1/6 \nonumber \\
B_3(x) &= x^3 - 3x^2/2 + x/2 \nonumber \\
B_4(x) &= x^4-2x^3+x^2-1/30 \label{bernoullipol}\\
B_5(x) &= x^5 - 5x^4/2 + 5x^3/3 - x/6 \nonumber \\
B_6(x) &= x^6-3x^5+5x^4/2-x^2/2+1/42 \nonumber \\
B_7(x) &= x^7 - 7x^6/2 +  7x^5/2 - 7x^3/6 + x/6 \nonumber \\
B_8(x) &= x^8-4x^7+14x^6/3 - 7x^4/3 +2x^2/3-1/30. \nonumber
\end{align}
Only degrees $n \le 8$ are programmed for now.
\endtab
\code

double Harmonic (long n);
\endcode
\tab Computes the $n$-th harmonic number $H_n  = \sum_{j=1}^n 1/j$.
\endtab
\code

double Harmonic2 (long n);
\endcode
\tab Computes the sum
\[
\sideset{}{'}\sum_{-n/2<j\le n/2}\; \frac 1{|j|},
\]
 where the symbol $\sum^\prime$ means that the term with $j=0$ is excluded
 from the sum.
\endtab
\code

double FourierC1 (double x, long n);
\endcode
\tab Computes and returns the value of the series (see \cite{vJOE92b})
\[
S(x, n) = \sum_{j=1}^{n} \frac{\cos(2\pi j x)}{j}.
\]
Restrictions: $n>0$ and $0 \le x \le 1$.
\endtab
\code

double FourierE1 (double x, long n);
\endcode
\tab Computes and returns the value of the series
\[
G(x, n) = \sideset{}{'}\sum_{-n/2<h\le n/2}\;  \frac{e^{2\pi i h x}}{|h|},
\]
 where the symbol $\sum^\prime$ means that the term with $h=0$ is excluded
 from the sum, and assuming that the imaginary part of $G(x, n)$ vanishes.
 Restrictions: $n>0$ and $0 \le x \le 1$.
\endtab
\code
}
\hide
#endif
\endhide
\endcode
