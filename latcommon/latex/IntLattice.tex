% Copyright (c) 2012 Richard Simard, Pierre L'Ecuyer, Université de Montréal.
% 
% This file is part of Lattice Builder.
% 
% Lattice Builder is free software: you can redistribute it and/or modify
% it under the terms of the GNU General Public License as published by
% the Free Software Foundation, either version 3 of the License, or
% (at your option) any later version.
% 
% Lattice Builder is distributed in the hope that it will be useful,
% but WITHOUT ANY WARRANTY; without even the implied warranty of
% MERCHANTABILITY or FITNESS FOR A PARTICULAR PURPOSE.  See the
% GNU General Public License for more details.
% 
% You should have received a copy of the GNU General Public License
% along with Lattice Builder.  If not, see <http://www.gnu.org/licenses/>.

\defmodule {IntLattice}

This class offers tools to manipulate lattice bases.
Each lattice is represented by a basis $V$ and its dual $W$.
It is sometimes possible, as in the case of lattices associated with
LCGs (linear congruential generators) or MRGs (multiple recursive generators),
 to multiply a lattice (and its dual) by a constant factor $m$
in such a way that they are included in $\mathbb{Z}^t$, allowing
exact representation of basis vector coordinates.
The duality relation will then read $V_i\cdot W_j = m\delta_{ij}$
for some integer constant {$m$}.

%%%%%%%%%%%%%%%%%%%%%%%%%%%%%%%%%%%%%%%%%%%%%%%%%%%%%%%%%%%%%
\bigskip
\hrule
\code\hide
#ifndef INTLATTICE_H
#define INTLATTICE_H
\endhide
#include "latcommon/Types.h"
#include "latcommon/Const.h"
#include "latcommon/Base.h"
#include "latcommon/Normalizer.h"
#include "latcommon/CoordinateSets.h"
#include <string>


namespace LatCommon {

class IntLattice {
public:

   IntLattice (const MScal & m, int k, int maxDim, NormType norm = L2NORM);
\endcode
\tabb
  Constructor. The modulus is initialized to $m$, the maximal dimension of the
  bases is set to \texttt{maxDim}, and the order is set to $k$.
\endtabb
\code

   IntLattice (const IntLattice & Lat);
\endcode
\tabb
  Copy constructor. The maximal dimension of the
  created basis is set equal to \texttt{Lat}'s current dimension.
\endtabb
\code

   IntLattice & operator= (const IntLattice & Lat);
\endcode
\tabb
 Assigns \texttt{Lat} to this object. The maximal dimension of this
  basis is set equal to \texttt{Lat}'s current dimension.
\endtabb
\code

   virtual ~IntLattice();
\endcode
\tabb
Destructor.
\endtabb
\code

   void copy (const IntLattice & lattice);
\endcode
\tabb
Same as assignment operator above.
\endtabb
\code

   int getDim() const \hide { return m_v.getDim(); } \endhide
\endcode
\tabb
Returns actual dimension \texttt{Dim}.
\endtabb
\code

   void setDim (int d);
\endcode
\tabb
Sets actual dimension to $d$.
\endtabb
\code

   int getMaxDim() const \hide { return m_v.getMaxDim(); } \endhide
\endcode
\tabb
Returns the maximal dimension of the lattice.
\endtabb
\code

   virtual void incDim();
\endcode
\tabb
Increases dimension \texttt{Dim} by 1.
\endtabb
\code

   virtual void buildBasis (int d);
\endcode
\tabb
Builds the basis for the lattice in dimension \texttt{d}.
\endtabb
\code

   NormType getNorm() const \hide { return m_v.getNorm(); } \endhide
\endcode
\tabb
Returns the norm used by the lattice.
\endtabb
\code

   MScal getM() const \hide { return m_m; } \endhide
\endcode
\tabb
  Returns the value of the modulus $m$ of the recurrence (the number of points
  of the lattice).
\endtabb
\code

   NScal getM2() const \hide { return m_m2; } \endhide
\endcode
\tabb
Returns the square number of points in the lattice.
\endtabb
\code

   int getOrder() const \hide { return m_order; } \endhide
\endcode
\tabb
Returns the order.
\endtabb
\code

   Normalizer * getNormalizer (NormaType norma, int alpha);
\endcode
\tabb
Creates and returns the normalizer corresponding to criterion \texttt{norma}.
In the case of the $P_\alpha$ test, the argument \texttt{alpha} = $\alpha$.
In all other cases, it is unused.
\endtabb
\code

   virtual std::string toStringCoef() const;
\endcode
\tabb
Returns the vector of multipliers (or coefficients) $A$ as a string.
\endtabb
\code

   Base & getPrimalBasis() \hide { return m_v; } \endhide
\endcode
\tabb
Returns the primal basis \texttt{V}.
\endtabb
\code

   Base & getDualBasis() \hide { return m_w; } \endhide
\endcode
\tabb
Returns the dual basis \texttt{W}.
\endtabb
\code

   void calcLgVolDual2 (bool dualF);
\endcode
\tabb Computes the logarithm of the normalization factor
 \texttt{m\_lgVolDual2} for the merit  in all dimensions %$\le $ \texttt{maxDim}
 for the dual lattice if \texttt{dualF} is \texttt{true}, and for the primal
 lattice if \texttt{dualF} is \texttt{false}.
\endtabb
\code

   double getLgVolDual2 (int i) const \hide { return m_lgVolDual2[i]; } \endhide
\endcode
\tabb
\endtabb
\code

   bool getXX (int i) const \hide { return m_xx[i]; } \endhide
\endcode
\tabb
\endtabb
\code

   void setXX (bool val, int i) \hide { m_xx[i] = val; } \endhide
\endcode
\tabb
\endtabb
\code

   void sort (int d);
\endcode
\tabb
Sorts the basis vectors with indices from $d+1$ to the dimension of the basis
   by increasing length. The dual vectors are permuted accordingly.
   Assumes that the lengths of the corresponding basis vectors
% \texttt{VV} and \texttt{WW}
   are up to date.
\endtabb
\code

   void permute (int i, int j);
\endcode
\tabb
Exchanges vectors $i$ and $j$ in the basis and in the dual basis.
\endtabb
\code

   void write (const char *filename) const;
\endcode
\tabb
Writes this basis in file named \texttt{filename}.
\endtabb
\code

   void trace (char* msg);
\endcode
\tabb
For debugging purposes.
\endtabb
\code

   void write (int flag);
\endcode
\tabb
Writes this basis on standard output. If \texttt{flag} $ > 0$, the norm of
the basis vectors will be recomputed; otherwise not.
\endtabb
\code

   void dualize();
\endcode
\tabb
Exchanges basis $V$ and its dual $W$.
\endtabb
\code

   bool checkDuality();
\endcode
\tabb
   Checks that the bases satisfy the duality relation
   $V[i]\cdot W[j] = m\,\delta_{ij}$. If so, returns \texttt{true}, otherwise
   returns \texttt{false}.
\endtabb
\code

   bool baseEquivalence (IntLattice & Lat);
\endcode
\tabb
   Checks that \texttt{Lat}'s basis and this basis are equivalent.
   If so, returns \texttt{true}, otherwise returns \texttt{false}.
\endtabb
\code

   void buildProjection (IntLattice* lattice, const Coordinates & proj);
\endcode
\tabb
Builds the basis (and dual basis) of the projection \texttt{proj} for this
lattice.  The result is placed in the \texttt{lattice} lattice. The basis is
triangularized to form a proper basis.
\endtabb
\code


protected:

   void init();
\endcode
\tabb
\endtabb
\code

   void kill();
\endcode
\tabb
Cleans and releases all the memory allocated for this lattice.
\endtabb
\code

   int m_order;
\endcode
\tabb
The order of the basis.
\endtabb
\code

   MScal m_m;
   NScal m_m2;
\endcode
\tabb
Number of points per unit volume (\texttt{m\_m}) and its square (\texttt{m\_m2}).
\endtabb
\code

   double m_lgm2;
\endcode
\tabb
  The logarithm $\log_2 (m^2)$.
\endtabb
\code

   Base m_v;
\endcode
\tabb
Primal basis of the lattice.
\endtabb
\code

   Base m_w;
\endcode
\tabb
Dual basis of the lattice.
\endtabb
\code

   double *m_lgVolDual2;
\endcode
\tabb
\endtabb
\code

   bool *m_xx;
   MScal m_t1, m_t2, m_t3;
   BMat m_vSI;
   Base m_vTemp;
\endcode
\tabb
Work variables.
\endtabb
\code
};

}
\hide
#endif
\endhide
\endcode
