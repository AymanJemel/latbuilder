% Copyright (c) 2012 Richard Simard, Pierre L'Ecuyer, Université de Montréal.
% 
% This file is part of Lattice Builder.
% 
% Lattice Builder is free software: you can redistribute it and/or modify
% it under the terms of the GNU General Public License as published by
% the Free Software Foundation, either version 3 of the License, or
% (at your option) any later version.
% 
% Lattice Builder is distributed in the hope that it will be useful,
% but WITHOUT ANY WARRANTY; without even the implied warranty of
% MERCHANTABILITY or FITNESS FOR A PARTICULAR PURPOSE.  See the
% GNU General Public License for more details.
% 
% You should have received a copy of the GNU General Public License
% along with Lattice Builder.  If not, see <http://www.gnu.org/licenses/>.

\defmodule {UtilLM}

This module describes various useful functions as well as
functions interfacing with NTL.

%%%%%%%%%%%%%%%%%%%%%%%%%%%%%%%%%%%%%%%%%%%%%%%%%%%%%%%%%%%%%%
\bigskip\hrule
\code\hide
#ifndef UTILLM_H
#define UTILLM_H
\endhide
#include <cstdlib>
#include <cmath>
#include <iostream>
#include <iomanip>
#include <sstream>
#include <string>

#include "NTL/tools.h"
#include "NTL/ZZ.h"
#include "latcommon/vec_lzz.h"
#include "latcommon/Types.h"
#include "latcommon/Const.h"



namespace LatMRG {

const double MAX_LONG_DOUBLE = 9007199254740992.0;
\endcode
\tab
Maximum integer that can be represented exactly as a
  \texttt{double}: $2^{53}$.
\endtab


%%%%%%%%%%%%%%%%%%%%%%%%%%%%%%%%%%%%%%%%%%%%%%%
\guisec{NTL compatibility utilities}
\code

inline long IsOdd (const long & x) \hide {
    return x & 1;
}\endhide
\endcode
\tab
Returns 1 if $x$ is odd, and 0 otherwise.
\endtab



%%%%%%%%%%%%%%%%%%%%%%%%%%%%%%%%%%%%%%%%%%%%%%%%%%%%%%%
\guisec{Mathematical functions}
\code

inline long power (long p, long i) \hide
{
   return NTL::power_long (p, i);
}
\endhide
\endcode
\tab
Returns $p^i$.
\endtab
\code


inline void power2 (long & z, long i) \hide
{
   z = NTL::power_long (2, i);
}
\endhide
\endcode
\tab
Sets $z = 2^i$.
\endtab
\code


inline double mysqrt (double x) \hide
{
    if (x < 0.0)
       return -1.0;
    return sqrt (x);
}
\endhide
\endcode
\tab
Returns $\sqrt{x}$ for $x\ge 0$, and $-1$ for $x < 0$.
\endtab
\code


inline double SqrRoot (double x) \hide
{
    return sqrt (x);
}
\endhide
\endcode
\tab
Returns $\sqrt{x}$. \richard{Cette fonction est-elle encore utilis\'ee?}
\endtab
\code


template <typename T>
inline double Log2 (const T & x)\hide
{
   return log (x) / 0.6931471805599453094;
}
\endhide
\endcode
\tab
Logarithm of $x$ in base 2.
\endtab
\code


inline double Log2 (long x)\hide
{
   return log ((double)x) / 0.6931471805599453094;
}
\endhide
\endcode
\tab
Logarithm of $x$ in base 2.
\endtab
\code


template <typename T>
inline long sign (const T & x)\hide
{
    if (x > 0) return 1; else if (x < 0) return -1; else return 0;
}
\endhide
\endcode
\tab
Returns 1, 0 or $-1$ depending on whether $x> 0$, $x= 0$ or $x< 0$
  respectively.
\endtab


% \newpage
%%%%%%%%%%%%%%%%%%%%%%%%%%%%%%%%%%%%%%%%%%%%%%%%
\guisec{Division and remainder}

For negative operands, the \texttt{/} and \texttt{\%} operators do not give the
same results for NTL large integers \texttt{ZZ} and for primitive types
\texttt{int} and \texttt{long}. The negative quotient differs by 1 and the
remainder also differs.
Thus the following small \texttt{inline} functions for division and remainder.
\richard{Pour certaines fonctions, les r\'esultats sont mis dans les premiers
arguments de la fonction pour \^etre compatible avec NTL; pour d'autres, 
ils sont mis dans les derniers arguments pour \^etre compatible avec notre
ancienne
version de \latmrg{} en Modula-2.
Plut\^ot d\'etestable. Je crois qu'il faudra un jour r\'earranger les arguments
des fonctions pour qu'elles suivent toutes la m\^eme convention que NTL.}
\code


inline void div (long & a, const long & b, const long & d)\hide
{
    a = b/d;
}
\endhide
\endcode
\tab
Integer division: $a = b/d$.
\endtab



%%%%%%%%%%%%%%%%%%%%%%%%%%%%%%%%%%%%%%%%%%%%%%%%%%%%%%%%%%%
\guisec{Vectors}

\code

template <typename Real>
inline void CreateVect (Real* & A, int d)\hide
{
    A = new Real[1 + d];
    for (int i = 0; i <= d; i++)
        A[i] = 0;
}
\endhide
\endcode
\tab
Allocates memory for the vector $A$ of dimensions
   $d+1$ and initializes its elements to 0.
\endtab
\code


template <typename Real>
inline void DeleteVect (Real* & A)\hide
{
    delete[] A;
//    A = 0;
}
\endhide
\endcode
\tab
Frees the memory used by the vector $A$.
\endtab
\code


template <typename Vect>
inline void CreateVect (Vect & A, int d)\hide
{
    A.SetLength (1 + d);
    // aaa clear (A);
}
\endhide
\endcode
\tab
Creates the vector $A$ of dimensions $d+1$
   and initializes its elements to 0.
\endtab
\code


template <typename Vect>
inline void DeleteVect (Vect & A)\hide
{
    A.kill ();
}
\endhide
\endcode
\tab
Frees the memory used by the vector $A$.
\endtab
\code


template <typename Real>
inline void SetZero (Real* A, int d)\hide
{
    for (int i = 0; i <= d; i++)
        A[i] = 0;
}
\endhide
\endcode
\tab
Sets components $[0..d]$ of $A$ to 0.
\endtab
\code


template <typename Real>
inline void SetValue (Real* A, int d, const Real & x)\hide
{
    for (int i = 0; i <= d; i++)
        A[i] = x;
}
\endhide
\endcode
\tab
Sets all components $[0..d]$ of $A$ to the value $x$.
\endtab
\code


inline void Invert (const MVect & A, MVect & B, int n)\hide
{
    conv(B[n], 1);
    for(int i = 0; i < n; i++){
       B[i] = -A[n - i];
    }
}
\endhide
\endcode
\tab
Transforms the polynomial $A_0 + A_1x^1 + \cdots + A_nx^n$ into
 $x^n - A_1x^{n-1} - \cdots - A_n$. The result is put in $B$.
\endtab
\code


template <typename Vect>
inline void CopyVect (const Vect & A, Vect & B, int n)\hide
{
    for (int k = 0; k <= n; k++)  B[k] = A[k];
}
\endhide
\endcode
\tab
Copies vector $A$ into vector $B$ using components $[0..n]$.
\endtab
\code


template <typename Xcal, typename Scal>
inline void ModifVect (Xcal *A, const Xcal *B, Scal x, int n)\hide
{
    Xcal a;
    conv (a, x);
    for (int i = 1; i <= n; i++)
        A[i] += B[i]*a;
}
\endhide
\endcode
\tab
Adds vector $B$ multiplied by $x$ to vector $A$ using components
$[1..n]$, and puts the result in $A$.
\endtab


%%%%%%%%%%%%%%%%%%%%%%%%%%%%%%%%%%%%%%%%%%%%%%%%%%%
\guisec{Matrices}

\code

template <typename Real>
inline void CreateMatr (Real** & A, int d)\hide
{
    A = new Real *[1 + d];
    for (int i = 0; i <= d; i++) {
        A[i] = new Real[1 + d];
        for (int j = 0; j <= d; j++)
            A[i][j] = 0; }
}
\endhide
\endcode
\tab
Allocates memory for the square matrix $A$ of dimensions
   $(d+1)\times(d+1)$. Initializes its elements to 0.
\endtab
\code


template <typename Real>
inline void DeleteMatr (Real** & A, int d)\hide
{
    for (int i = d; i >= 0; --i)
        delete[] A[i];
    delete[] A;
 //   A = 0;
}
\endhide
\endcode
\tab
Frees the memory used by the  $(d+1)\times(d+1)$ matrix $A$.
\endtab
\code


template <typename Real>
inline void CreateMatr (Real** & A, int line, int col)\hide
{
    A = new Real *[1+line];
    for (int i = 0; i <= line; i++) {
        A[i] = new Real[1+col];
        for (int j = 0; j <= col; j++)
            A[i][j] = 0;
    }
}
\endhide
\endcode
\tab
Allocates memory for the matrix $A$ of dimensions
   (\texttt{line} + 1) $\times$ (\texttt{col} + 1). Initializes its elements to 0.
\endtab
\code


template<typename Real>
inline void DeleteMatr (Real** & A, int line, int col)\hide
{
    for (int i = line; i >= 0; --i)
        delete [] A[i];
    delete [] A;
//    A = 0;
}
\endhide
\endcode
\tab
Frees the memory used by the matrix $A$.
\endtab
\code


inline void CreateMatr (MMat& A, int d)\hide
{
    A.SetDims (1 + d, 1 + d);
    //clear (A);
}
\endhide
\endcode
\tab
Creates the square matrix $A$ of dimensions $(d+1)\times(d+1)$
   and initializes its elements to 0.
\endtab
\code


inline void CreateMatr (MMatP & A, int d)\hide
{
    A.SetDims (1 + d, 1 + d);   clear (A);
}
\endhide
\endcode
\tab
As above.
\endtab
\code


inline void CreateMatr (MMat& A, int line, int col)\hide
{
    A.SetDims(1 + line, 1 + col);
    //clear (A);
}
\endhide
\endcode
\tab
Creates the matrix $A$ of dimensions
   (\texttt{line} + 1) $\times$ (\texttt{col} + 1). Initializes its elements to 0.
\endtab
\code


inline void CreateMatr (MMatP & A, int line, int col)\hide
{
    A.SetDims (1 + line, 1 + col);   clear (A);
}
\endhide
\endcode
\tab
As above.
\endtab
\code


inline void DeleteMatr (MMat& A)\hide
{
    A.kill ();
}
\endhide
\endcode
\tab
Deletes the matrix $A$.
\endtab
\code


inline void DeleteMatr (MMatP & A)\hide
{
    A.kill ();
}
\endhide
\endcode
\tab
As above.
\endtab
\code


template <typename Matr>
inline void CopyMatr (const Matr & A, Matr & B, int n)\hide
{
    for (int i = 1; i <= n; i++)
        for (int j = 1; j <= n; j++)
            B[i][j] = A[i][j];
}
\endhide
\endcode
\tab
Copies matrix $A$ into matrix $B$.
\endtab
\code


template <typename Matr>
inline void CopyMatr (const Matr & A, Matr & B, int line, int col)\hide
{
    for (int i = 1; i <= line; i++)
        for (int j = 1; j <= col; j++)
            B[i][j] = A[i][j];
}
\endhide
\endcode
\tab
As above.
\endtab
\code


template <typename MatT>
std::string toStr (const MatT & mat, int d1, int d2) \hide
{
   std::ostringstream ostr;
   for (int i = 1; i <= d1; i++) {
      ostr << "[";
      for (int j = 1; j <= d2; j++) {
         ostr << std::setprecision (2) << std::setw (6) << mat[i][j] <<
            std::setw (2) << " ";
      }
      ostr << "]\n";
   }
   return ostr.str ();
}\endhide
\endcode
\tab
Transforms \texttt{mat} into a string. Prints the first $d1$ rows and
 $d2$ columns. Indices start at 1. Elements with index 0 are not printed.
\endtab
\code

}     // namespace LatMRG
\endcode

%%%%%%%%%%%%%%%%%%%%%%%%%%%%%%%%%%%%%%%%%%%%%%%%%%%%%
\guisec{Some other compatibility utilities}

\code
namespace NTL {

inline void conv (long & l, const char* c) \hide {
    l = strtol(c, (char **) NULL, 10);
}\endhide
\endcode
\tab
Converts the array of characters (string) \texttt{c} into \texttt{l}.
\endtab
\code


inline void conv (double & r, const char* c) \hide {
    r = strtod(c, (char **) NULL);
}\endhide
\endcode
\tab
Converts the array of characters (string) \texttt{c} into \texttt{r}.
\endtab
\code

}     // namespace
\hide
#endif
\endhide
\endcode
